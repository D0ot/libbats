\chapter{Running the Simulation}

\section{Starting}
To start the simulation server, run the following:
\begin{verbatim}
$ rcssserver3d
\end{verbatim}
This will generate some output to the terminal, including a list of 'not found' and 'Unknow function' errors. These are actually more like warnings and can be ignored. Next, you will have to start a monitor to be able to see what is going on in the simulation. If you run the monitor on te same machine as the simulator, use:
\begin{verbatim}
$ rcssmonitor3d
\end{verbatim}
If you run the monitor on a different machine, which can boost the speed of the simulator if you experience lag, use:
\begin{verbatim}
$ rcssmonitor3d --server 192.168.1.2
\end{verbatim}
Of course, use the actual address of the machine you run the server on. You can connect as many monitors you want (and your machine(s) can handle).

For convenience, there is also a script that starts both the server and the monitor, and kills the server when you close the monitor:
\begin{verbatim}
$ rcsoccersim3d
\end{verbatim}

\section{Controls}

\begin{center}
	\begin{tabular}{ll}
		\keystroke{1}-\keystroke{7} & Move camera to default positions \\
		\LArrow, \UArrow, \DArrow, \RArrow & Move camera horizontally \\
		\PgUp, \PgDown & Move camera vertically \\
		Click + drag & Turn camera \\
		\keystroke{K} & Kick-off \\
		\keystroke{B} & Drop ball \\
		\keystroke{L} & Free kick for the left team \\
		\keystroke{R} & Free kick for the right team \\
		\keystroke{Q} & Quit \\
	\end{tabular}
	\label{tab:}
\end{center}
