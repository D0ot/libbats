\chapter{Introduction}

If you read this, you will probably already know about the RoboCup
initiaive: to progress robotics and artificial intelligence through
competition, to such a level that in 2050 an artificial football team
can beat the human world champions. Several leagues exist to focus on
seperate aspects of this problem, of which the 3D Soccer Simulation
(Sub) League is one. The aim of this league is to develop team
strategies and behaviors that are not (yet) feasible in the hardware
leagues.

This is the manual of \libbats, a library that can be used to get
started in the RoboCup 3D Simulation. It was originally developed by
the RoboCup 3D Simulation team the Little Green BATS in
2006. Currently it is being maintained and used in competitions by the
BATS and by team Bold Hearts, both vice world champion teams (in 2007
and 2009 respectively). This library can freely be used, both as in
free beer and free speech, to create your own (team of) RoboCup 3D
simulation agent(s) and to do research and enter competitions with. It
supplies some modules to get you started quickly, is generic enough to
add any algorithm, behavior model or learning method and still gives
you detailed control of the basics if required.

This manual is intended to give you an overview of the library and to
get you familiar with the structure and use of its parts. You can have
your own agent running within a few lines of code. More detailed
information on all possibilities can be found in the documentation in
the source code. If you want you can dig through all this code, but we
suggest to install doxygen
\footnote{\url{http://www.stack.nl/~dimitri/doxygen/}} instead. If you
do this and follow the installation instructions in chapter
\ref{chInstallation}, you will find this documentation nicely
formatted with HTML in the {\tt docs/html} directory.

The next chapter will give a brief discussion of the
SimSpark/RCSSServer3D simulation environment, which is used in the 3D
Soccer Simulation, and the way an agent should interact with it. This
chapter can in theory be skipped, since \libbats offers abstractions
and tools such that you don't have to worry about low level
issues. However, a general understanding of these will probably be
beneficial anyway.

After that we will give a tutorial of how to install the simulation
environment and \libbats, and some configuration options. Chapter
\ref{chQuickstart} takes you through the steps of creating your first
agent and shows how easy it is to set up a new team using
\libbats. The next chapters will go into the different parts and
modules of \libbats in more depth.

If something is still unclear, you found a bug or just have a question
related to this library, do not hesitate to contact us, preferably
through \libbats' Launchpad
page\footnote{\url{http://launchpad.net/littlegreenbats}}. Good luck
and happy coding!
